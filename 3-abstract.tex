% add to Table of content
\addcontentsline{toc}{part}{ABSTRACT}

% set 0 indentation
\setlength{\parindent}{0pt}
% set paragraph space = 1 space
\setlength{\parskip}{1em}
% set line space = 1.5
\setlength{\baselineskip}{1.5em}

\begin{center}
  \fontsize{14}{17}\selectfont{\textbf{
    ABSTRACT
  }}
\end{center}
\vspace{2em}

Drones have become inexpensive and can be used to survey an area of interest and send real time data to the ground control station. However, low-cost drones suffer from the constraints of limited flight time due to battery limitations. If the region is denied of network infrastructure (cellular network) such as in remote areas or disaster then the range of the drone is limited to the wifi network range. A system that uses multiple drones can be used to survey and acquire real-time data as it reduces the amount of flight time for each drone. The range of the drones can be extended using a wireless mesh network with each drone working as a node of the mesh. 

For such systems, the operator can either set the flight path for each drone before the mission or many operators can fly the drones individually.  However, it becomes the responsibility of the operators that the drones do not collide with each other, and they do not fly beyond the range of the mesh network. The mesh nodes are mobile, therefore the area of coverage of the network is dynamic throughout the mission.

In this study, I propose a centralized system called Pegasus for autonomous exploration and aerial map building with multiple drones using a wireless mesh network. The different phases in the pipeline used in this system to achieve autonomous flight and map building are described.  The pipeline consists of the presentation layer, multi-agent coverage path planning layer, motion control layer, real time image acquisition layer, and map generation layer using structure from motion. To evaluate the system, a simulation framework is presented in this study. Finally, the system is evaluated in the real world with a single drone. 


\textbf{Keywords:} Multi-agent CPP, Wireless mesh network, SfM, Autonomous UAV