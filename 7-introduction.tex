% set 0.5 inch indentation
\setlength{\parindent}{0pt} 
% set paragraph space = 0 space
\setlength{\parskip}{0mm}
% set line space 1.5
\setlength{\baselineskip}{1.6em}

\chapter{INTRODUCTION} 

\section{Background of the Study}

The availability and financial accessibility of unmanned aerial vehicles (UAVs) have made them more widespread, with applications ranging over a wide range of civilian activities. Multi-rotors are used for recreational flying, research, cinematography, disaster observation, logistics, agriculture,  public safety, construction, surveillance, and environmental protection, amongst other purposes. 

A cluster of inexpensive autonomous multi-rotors connected through a wireless mesh network that can be deployed in a post-disaster situation could help avoid dangerous situations faced by ground-based human observers in such scenarios. The drones should be able to perform aerial mapping quickly, as they can coordinate so that one drone does not need to visit locations already visited by other drones in the cluster. The range of the cluster can also be large, since each drone would act as a wireless mesh access point. At the same time, each drone could act as an access point to provide connectivity in the disaster-stricken area.  

Aerial mapping drones can be outfitted with an array of sensors such as ultrasonic sensors, infrared sensors, multi-array laser sensors, and RGB-D cameras. However, systems with many sensors will suffer from integration complexity, weight constraints, reduced battery lifetime, and high cost. Structure from motion (SfM) using monocular cameras attached to each drone provides a low cost, lightweight, simple alternative to sensor-intensive approaches. 

This study proposes a system for command and control of a cluster of autonomous drones, each being a node in a wireless mesh network, controlled from a centralized control station that oversees coordinated exploratory path planning and aerial mapping using SfM.

\section{Statement of the Problem}

An inexpensive cluster of UAVs providing autonomous coordinated aerial mapping after a disaster can be a valuable asset for disaster observation and public safety that does not pose any risks for ground-based surveillance personnel. Multiple drones will reduce flight time and increase the spatial extent of the SfM result. A wireless mesh network can also be an inexpensive solution when communication infrastructure such as cellular data services are dysfunctional. Increasing the number of drones in the system can enable increased range of the mesh network, because each drone acts as a mesh node. Providing emergency network services using UAVs and mesh networks in disaster-stricken areas is an active research area, as discussed by \shortciteA{Chand2018}, but the best method to combine coverage for mapping while maintaining mesh connectivity is an open problem. \shortciteA{Sabino2018} discuss optimal placement of UAVs in a mesh network, and separately, a survey of the multi-agent coverage path planning problem (CPP) by \shortciteA{Almadhoun2019} reviews many CPP methods proposed in the literature. However, the requirement for mesh network connectivity imposes severe constraint on the movement of agents, making the resulting "constrained CPP" an open research area.

Simulation environments exist that can be used for multi-drone planning and wireless mesh networks. Pixhawk PX4's software in the loop (SITL) along with Gazebo can simulate a physical fleet of UAVs with cameras. Wireless mesh networks can be simulated by Network Simulator 3 (NS3). The Robot Operating System (ROS), with its range of tools and available packages, provides a strong platform to develop this system. FlySimNet by \shortciteA{Baidya2018} is a UAV network simulator created by combining NS3 and Ardupilot's UAV SITL simulator with lightweight publish/subscribe (ZMQ) based middleware. It uses a custom graphical ground control system (GCS) application that communicates via the ZMQ messaging service, which does not support ROS, and cameras are not supported by Ardupilot SITL. The CORNET middleware framework by \shortciteA{AcharyaRBCCPS} combines Gazebo, NS3 and PX4, but it is available only as a published paper and implementation of their method is not quite clear. Neither of these frameworks implement wireless mesh network simulation on NS3. Gazebo, NS3, PX4, and ROS together provide an open source platform, but integration of these tools is required.

This thesis aims to design and implement a multi-agent CPP method that satisfies wireless mesh network constraints, runs multi-agent SfM. It also aims to develop a simulation framework to support the CPP method.

\section{Objectives of the Study}

The main objective of this thesis is to improve on the state of the art in mapping disaster-stricken areas by designing and implementing a method for autonomous control of a multi-agent exploration and coordinated aerial mapping team performing SfM. The focus of this study is to map a disaster-stricken area with the possible damage to communication infrastructure; therefore, the system will use a wireless mesh network for internal communication, with each agent working as a mesh node. The objectives can be decomposed into the following specific tasks:
\begin{itemize}
	\item Design and implement a user-friendly pipeline utilizing the Robot Operating System (ROS) that allows the operator to select an area to survey.
	
	\item Design and implement strategies to find the transformations between the global map and the local map of each drone continuously such that the control station can coordinate the position of each drone with respect to the global map.
	
	\item Design and implement a method for establishing a reliable communication channel between each drone and the control station over the mesh network.
	
	\item Design and implement an autonomous exploration and coordination system for the agents that operates within the restriction of the range of the wireless mesh network. The system should maximize the extent of the mappable area, by planning an appropriate arrangement of agents over time.
	
	\item Implement a multi-agent SfM method capable of generating an aerial map of the area under surveillance.
	
	\item Design and implement a simulation framework for the entire system using PixHawk PX4 as the drone flight controller, Gazebo and Network Simulator 3 as simulation tools, ROS for overall coordination, and useful robotics/vision algorithms.
	
	\item Design and implement a wireless mesh network with each drone as a mesh node. As the drones are fast-moving agents, this will require the study and implementation of solutions to maintain a reliable network with a usable quality of service, despite constant topology changes.
	
	\item Evaluate the system's ability to execute effectively in the real world.
\end{itemize}

\section{Limitations and Scope}

The study will not cover the following.
\begin{itemize}
	\item \textit{Obstacle detection and avoidance}: The control station will generate paths for the drones, such that they will not collide with each other. It is assumed that no unknown obstacle will be present at the drone's operational altitude. It is the operator's responsibility to indicate the minimum altitude during mapping.
	
	\item \textit{Dynamic path reconfiguration}: The control station will generate complete paths for the drones before the start of the mission.
	
	\item \textit{Recovery mechanism}: If communication is lost with any drone, it will return to the home position using the default behavior in PX4. The remaining drones will continue their operation.
	
	\item \textit{Global Positioning System failure}: It is assumed that GPS will be continuously functional.
	
	\item \textit{Number of drones in evaluation}: The simulation environment have been tested with a maximum of three drones, real world validation uses only one drone.
\end{itemize}

\section{Organization of the Thesis}

The thesis is organized as follows:
\begin{itemize}
	\item Chapter \ref{ch:literature-review} provides a literature review related to this study.
	\item Chapter \ref{ch:methodology} proposes the methodology for the system developed in this study.
	\item Chapter \ref{ch:results} presents experimental results using the system developed in this study.
	\item Finally, chapter \ref{ch:conclusion} concludes the thesis and provides recommendation for future work.
\end{itemize}

\FloatBarrier
