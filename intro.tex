\setlength{\footskip}{8mm}

\chapter{Introduction}

\textit{  }

\section{Overview}

The availability and financial accessibility of unmanned aerial vehicles (UAVs) have made them more widespread, finding application in a wide range of civilian activities. Multi-rotors are used for recreational flying, research, cinematography, disaster observation, logistics, agriculture,  public safety, construction, surveillance, and environmental protection, amongst other things. 

A cluster of inexpensive autonomous multi-rotors connected through a wireless mesh network that can be deployed in a post-disaster situation can help avoid dangerous situations faced by ground-based human observers in such scenarios. The drones can do aerial mapping quickly, as they can coordinate so that one drone does not need to visit locations already visited by other drones in the cluster. The range of the cluster can also be large, as each drone will act as a wireless mesh access point. Furthermore, each drone can act as an access point to provide connectivity in the area.  

For aerial mapping, drones can be fitted with an array of sensors such as ultrasonic, infrared, multi-array laser sensors, and RGB-D cameras. Systems with many sensors have integration complexities and may suffer from weight constraints, reduced battery lifetime, and high cost. Monocular visual SLAM provides a low cost, light weight, simple alternative to sensor-intensive approaches. 

This study proposes a system of a cluster of autonomous drones, each being a node in a wireless mesh network, that performs coordinated exploratory aerial mapping using vSLAM.

\section{Problem Statement}


An inexpensive cluster of UAVs providing autonomous coordinated aerial mapping after a disaster can be a valuable asset for disaster observation and public safety, without posing risk for ground-based surveillance personnel. A wireless mesh network can also be an inexpensive solution when communication infrastructure like cellular data service is dysfunctional. The number of drones in the system is proportional to the desired range of the mesh network, because each drone can be a mesh node. UAVs providing emergency network services using mesh networks in disaster-struck areas is an active research area as discussed by  \shortciteA{dronewirelessmeshnetworkchand}. \shortciteA{Sabino_2018} discusses the optimal placement of UAVs' in a mesh network. A survey by \shortciteA{ZOU2019461} discusses the various multi-agent vSLAM methods available. \shortciteA{multiuavnetwork} uses mesh networks and SLAM with multiple UAVs but they control the UAVs manually, the SLAM results are not merged into a global map, and they do not implement coordination and autonomous path planning within the constraints of a mesh network.

Monocular vSLAM requires only one camera, but it suffers from issues of map initialization and scale ambiguity. That is, the size of the environment as mapped will not be to scale. Integrating accelerometer and gyroscope measurements can help determine an approximate scale.


\section{Objectives}


The main objective of this thesis is to improve on the state of the art in mapping disaster-stricken areas by implementing a method for autonomous multi-agent exploration and coordinated aerial mapping with vSLAM. The focus of this study is to map a disaster-struck area with the possibility of damage to communication infrastructure; therefore, the system will use a wireless mesh network for internal communication, with each agent working as a mesh node. The objectives can be decomposed into the following specific tasks:
\begin{itemize}
	\item Design and implement a wireless mesh network with each drone as a mesh node. As the drones are fast-moving agents, this will require study and implementation of solutions to maintain a reliable network with a usable quality of service, despite constant topology changes.
	\item Implement monocular vSLAM to generate an aerial map of the area under surveillance, and merge the point clouds generated by different agents into a single global map.
	\item Design and implement an autonomous exploration and coordination system for the agents that operates within the restriction of the range of the wireless mesh network. This system should also maximize the range of the surveillance, by finding an optimal arrangement of the agents over time.
\end{itemize}

\todo{ Flight time restrictions }




\section{Limitations and Scope}

Some text ...

\section{Thesis Outline}

I organize the rest of this dissertation as follows.

In Chapter \ref{ch:literature-review}, I describe the literature review.

In Chapter \ref{ch:methodology}, I propose my methodology.

In Chapter \ref{ch:results}, I present the experimental results.

Finally, in Chapter \ref{ch:conclusion}, I conclude my thesis.

\FloatBarrier
